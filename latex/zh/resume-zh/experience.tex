%-------------------------------------------------------------------------------
%	SECTION TITLE
%-------------------------------------------------------------------------------
\cvsection{工作经历}

%-------------------------------------------------------------------------------
%	CONTENT
%-------------------------------------------------------------------------------
\begin{cventries}

%---------------------------------------------------------
% 定义后端工程师职位的简历条目
  \cventry
    {后端工程师} % Job title
    {美钛技术服务(上海)有限公司广州分公司} % Organization
    {中国广州} % Location
    {2025年2月 - 至今} % Date(s)
    {
      \begin{cvitems} % Description(s) of tasks/responsibilities
        \item {通过TEKsystems外派至汇丰银行,为汇丰企业科技部的财务转型平台开发和优化后端系统。}
        \item {维护财务数据处理功能模块,包括数据导入、验证和导出流程,同时优化财务科目提交与审批工作流。运用会计学、总账和银行系统专业知识提升系统效能。}
        \item {负责从本地开发到用户验收测试及生产环境部署的全生命周期管理,解决死锁问题及IBM Db2存储过程故障。深度使用Copilot完成根因分析、Python脚本编写等复杂任务,撰写的约50篇技术文档成为团队应对同类问题的重要参考。}
        \item {技术栈涉及Java、Spring Boot、IBM Db2、多线程、Maven、Nexus、Angular和Python,配套工具包括Windows、Control-M、IBM WebSphere应用服务器及Copilot。}
      \end{cvitems}
    }

%---------------------------------------------------------
  \cventry
    {自由职业者} % Job title
    {个人执业} % Organization
    {中国广州} % Location
    {2023年8月 - 2025年1月} % Date(s)
    {
      \begin{cvitems} % Description(s) of tasks/responsibilities
        \item {分析并重构约30个机器学习项目的核心组件,项目来源包括PyTorch/TensorFlow教程、Coursera课程及开源仓库。获得两项Coursera认证:机器学习专项课程和深度学习专项课程。}
        \item {备考专升本课程,主攻高等数学、计算机组成原理和线性代数。通过日本音乐和TikTok视频沉浸式学习日语。}
        \item {作为全栈开发者参与AI故事机器人开发,基于Claude API实现个性化叙事功能。使用Python/Flask/React/Nginx构建提示配置系统和管理界面,部署于AWS云平台,采用Prometheus监控和ELK日志管理,ChatGPT-4辅助编码。}
        \item {维护原创技术博客(\url{https://lzwjava.github.io},431篇),运用大语言模型实现9语种翻译、文本转语音及PDF/EPUB格式支持。集成GitHub工作流,使用LaTeX排版学术论文和简历。技术栈涵盖Python/Jekyll/Deepseek/Mistral。}
        \item {实验研究llama.cpp/嵌入模型/重排序/RAG架构及MMLU基准测试,集成Jina AI/Tavily AI Search/ElevenLabs等搜索引擎API。}
      \end{cvitems}
    }
%---------------------------------------------------------
  \cventry
    {后端工程师} % Job title
    {深圳法本信息技术有限公司} % Organization
    {中国广州} % Location
    {2022年11月 - 2023年7月} % Date(s)
    {
      \begin{cvitems} % Description(s) of tasks/responsibilities
        \item {法本是中国领先的软件技术服务商,服务对象汇丰PayMe是面向香港居民的移动支付平台。}
        \item {参与PayMe自动充值功能后端开发:监控Azure EventHub支付事件,当用户余额不足时自动从绑定的信用卡/借记卡扣款。采用面向对象编程优雅处理业务逻辑,通过AOP实现充值配置变更的审计日志。}
        \item {完成公司AWS培训后主导API云迁移改造:实现基于请求头的路由策略,确保数据库安全配置,参与微服务云原生部署。}
        \item {技术栈包括Java/Spring/Kafka,云服务采用Azure/Azure DevOps/AWS。}
      \end{cvitems}
    }

%---------------------------------------------------------
  \cventry
    {后端工程师} % Job title
    {北京博彦科技股份有限公司} % Organization
    {中国广州} % Location
    {2021年12月 - 2022年11月} % Date(s)
    {
      \begin{cvitems} % Description(s) of tasks/responsibilities
        \item {外派至东南亚最大银行星展银行,参与DBS Client Connect(股票交易系统)和DBS DigiBank CN(数字银行)项目开发。}
        \item {在股票交易微服务中实现股票展示、客户信息、交易前检查及订单提交功能,集成Avaloq API提升底层能力,运用编辑距离算法优化股票代码搜索体验。}
        \item {在数字银行项目开发共同基金、结构性理财产品等微服务,通过Pivotal Cloud Foundry日志分析生成QPS报告。开发Karate测试自动化工具提升测试覆盖率。}
        \item {采用Java/Spring Cloud技术栈,基于Pivotal云平台和Kibana日志系统,实践BDD/TDD方法论。}
      \end{cvitems}
    }

%---------------------------------------------------------
\cventry
{自由职业者} % Job title
{个人执业} % Organization
{中国广州} % Location
{2020年1月 - 2021年11月} % Date(s)
{
  \begin{cvitems} % Description(s) of tasks/responsibilities
    \item {撰写技术博客分享知识,通过Netflix和英文原著提升语言能力,解决约500道算法题并参加Codeforces竞赛。}
    \item {实践Spark/Hadoop/Kubernetes/Docker入门教程,积累大数据和云原生技术基础。}
    \item {承接LED显示屏官网开发(lvchensign.com)、ShowMeBug企业微信集成、无纺布贸易数据爬虫、mathjax2mobi电子书工具等自由项目。}
    \item {企业微信集成项目使用Ruby on Rails实现技术面试工具对接,数据爬虫项目通过Python/Selenium自动化采集并生成商业分析报告。}
    \item {mathjax2mobi工具将含LaTeX公式的HTML转换为SVG图片,支持MOBI等电子书格式,技术栈含Python/BeautifulSoup/Selenium。}
  \end{cvitems}
}

%---------------------------------------------------------
\cventry
{创始人兼全栈工程师} % Job title
{北京平方根科技有限公司} % Organization
{中国北京} % Location
{2016年7月 - 2019年12月} % Date(s)
{
  \begin{cvitems} % Description(s) of tasks/responsibilities
    \item {运营知识直播平台"趣直播"(2016-2017),累计举办80场讲座,获3万用户和百万级浏览量。技术栈:PHP/Vue/MySQL/阿里云/微信SDK。}
    \item {转型软件咨询业务(2018-2019),完成50个定制项目,总营收300万元。主导项目包括:}
    \item {面包Live后端重构:将ThinkPHP/Node.js/Go技术栈统一迁移至Laravel,实现课程、支付、分销等模块,与喜马拉雅平台内容同步。}
    \item {江苏卫视《最强大脑》微信小程序:全栈开发益智答题游戏,通过Redis应对高并发场景。}
    \item {冲顶大会直播答题APP:基于Java/Spring/Kafka技术栈实现实时问答系统,运用Socket.IO处理即时交互,研究SEI时间戳同步方案保障直播流与题目同步。}
  \end{cvitems}
}

%---------------------------------------------------------
\cventry
{联合创始人兼全栈工程师} % Job title
{北京大米互娱有限公司} % Organization
{中国北京} % Location
{2015年11月 - 2016年7月} % Date(s)
{
  \begin{cvitems} % Description(s) of tasks/responsibilities
    \item {创立专业代码评审平台CodeReview,实现代码提交、专家评审、活动管理等功能,积累3000名用户。}
    \item {负责后端系统及半数前端开发,技术栈包括PHP/Vue/CodeIgniter/阿里云/Ping++支付。}
  \end{cvitems}
}

%---------------------------------------------------------
\cventry
{软件工程师} % Job title
{美味书签(北京)信息技术有限公司(LeanCloud)} % Organization
{中国北京} % Location
{2014年7月 - 2015年11月} % Date(s)
{
  \begin{cvitems} % Description(s) of tasks/responsibilities
    \item {参与LeanCloud Objective-C/Java SDK开发,主导即时通讯演示应用LeanChat的iOS/Android客户端开发。}
    \item {技术栈涵盖iOS/Android SDK、Cocoapods、Xcode、Android Studio及Angular框架。}
  \end{cvitems}
}

%---------------------------------------------------------
\end{cventries}