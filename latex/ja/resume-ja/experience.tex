%-------------------------------------------------------------------------------
%	セクションタイトル
%-------------------------------------------------------------------------------
\cvsection{職務経歴}

%-------------------------------------------------------------------------------
%	内容
%-------------------------------------------------------------------------------
\begin{cventries}

%---------------------------------------------------------
% バックエンドエンジニア職のCVエントリ定義
  \cventry
    {バックエンドエンジニア} % 職名
    {美泰科技服務(上海)有限公司広州支社} % 組織
    {中国・広州} % 場所
    {2025年2月 - 現在} % 日付
    {
      \begin{cvitems} % 業務内容/責任の説明
        \item {TEKsystemsを通じてHSBC銀行に派遣され、企業技術部門の財務変革プラットフォーム向けバックエンドシステムの開発・最適化を担当。}
        \item {財務データ処理(インポート・検証・エクスポート)機能の保守と、新規財務ヘッダーの提出・承認ワークフローの改善を実施。会計・元帳・銀行システムに関する専門知識を活用。}
        \item {ローカル開発からUATテスト、本番リリースまでのフル開発ライフサイクルを管理。デッドロックやIBM Db2プロシージャなどの問題解決を実施。Copilotを活用して根本原因分析、Pythonスクリプト生成、約50件の技術ガイド作成を効率化。}
        \item {Java、Spring Boot、IBM Db2、マルチスレッド、Maven、Nexus、Angular、Pythonなどの技術と、Windows、Control-M、IBM WebSphere Application Server、Copilotなどのツールを使用。}
      \end{cvitems}
    }

%---------------------------------------------------------
  \cventry
    {フリーランス} % 職名
    {個人事業} % 組織
    {中国・広州} % 場所
    {2023年8月 - 2025年1月} % 日付
    {
      \begin{cvitems} % 業務内容/責任の説明
        \item {PyTorch、TensorFlowチュートリアル、Courseraコース、オープンソースリポジトリから約30の機械学習プロジェクトのコアコンポーネントを分析・再実装。Courseraで「Machine Learning Specialization」と「Deep Learning Specialization」の修了証を取得。}
        \item {高等数学、計算機組織、線形代数に焦点を当てた短期大学士取得のための試験準備。日本語の音楽やTikTok動画を通じて日本語を学習。}
        \item {Claude APIを活用したAIストリーボットのフルスタック開発に貢献。Python、Flask、React、Nginxを使用し、AWSにデプロイ。Prometheusによる監視、ELKスタックによるログ管理、ChatGPT-4によるコーディング支援を実施。}
        \item {技術ブログ(\url{https://lzwjava.github.io})を431記事で運営。大規模言語モデルを用いて9言語への翻訳、テキスト音声変換、PDF/EPUB形式対応を実現。GitHubワークフローを統合し、LaTeXで学術論文や履歴書を作成。}
        \item {llama.cpp、埋め込みモデル、リランカー、検索拡張生成(RAG)、MMLUベンチマークの実験を実施。Jina AI、Tavily AI Search API、ElevenLabs APIなどの検索エンジンAPIを統合。}
      \end{cvitems}
    }
%---------------------------------------------------------
  \cventry
    {バックエンドエンジニア} % 職名
    {深圳市法本信息技術有限公司} % 組織
    {中国・広州} % 場所
    {2022年11月 - 2023年7月} % 日付
    {
      \begin{cvitems} % 業務内容/責任の説明
        \item {法本を経由してHSBC銀行のPayMeプロジェクトに参画。自動チャージ機能のバックエンド開発を担当。ユーザー残高が一定額を下回った際にクレジット/デビットカードから自動入金するシステムを開発。}
        \item {Azure EventHubからの支払い後イベントを監視し、設定変更時に自動チャージをチェック。オブジェクト指向プログラミングでケースを適切に処理し、アスペクト指向プログラミングで監査ログを実装。}
        \item {AWSトレーニング参加後、クラウド移行に積極的に貢献。リクエストヘッダーベースのルーティングAPIをリファクタリングし、マイクロサービスを新規クラウドインフラにデプロイ。}
        \item {Java、Spring、Kafkaと、Azure、Azure DevOps、AWSなどのクラウドサービスを活用。}
      \end{cvitems}
    }

%---------------------------------------------------------
  \cventry
    {バックエンドエンジニア} % 職名
    {博彦科技股份有限公司} % 組織
    {中国・広州} % 場所
    {2021年12月 - 2022年11月} % 日付
    {
      \begin{cvitems} % 業務内容/責任の説明
        \item {DBS銀行のDBS Client ConnectとDBS DigiBank CNプロジェクトに参画。}
        \item {株式取引マイクロサービスの開発に貢献。Avaloq APIを統合し、編集距離アルゴリズムで株コード検索を改善。}
        \item {投資信託・構造化商品・ポートフォリオ管理マイクロサービスの開発に参加。Pivotal Cloud Foundryのログ分析でパフォーマンステストを支援。Karate向けテスト自動生成ツールを開発。}
        \item {Java、Spring Cloud、Jira、Jenkins、Pivotal Cloud Foundry、Kibanaなどの技術を活用。BDDとTDD手法を適用。}
      \end{cvitems}
    }

%---------------------------------------------------------
\cventry
{フリーランス} % 職名
{個人事業} % 組織
{中国・広州} % 場所
{2020年1月 - 2021年11月} % 日付
{
  \begin{cvitems} % 業務内容/責任の説明
    \item {技術ブログを執筆・公開。Netflixや文学作品を通じて英語力を向上。約500のアルゴリズム問題に取り組み、Codeforcesコンテストに参加。}
    \item {Spark、Hadoop、Kubernetes、Dockerの入門チュートリアルを実施。}
    \item {LED看板ウェブサイト開発、ShowMeBugの企業微信統合、貿易データ収集スクレイパー、eBookツールmathjax2mobiなどのプロジェクトを遂行。}
    \item {Bootstrap、HTML、JavaScriptでLED看板製造会社のウェブサイトを開発。}
    \item {Ruby on RailsとWeChat SDKで技術面接ツールの統合を実現。}
    \item {PythonとSeleniumで不織布会社向け貿易データスクレイパーを開発。}
    \item {MathJax数式をSVG画像に変換するeBookツールmathjax2mobiを開発。}
  \end{cvitems}
}

%---------------------------------------------------------
\cventry
{創業者兼フルスタックエンジニア} % 職名
{北京平方根科技有限公司} % 組織
{中国・北京} % 場所
{2016年7月 - 2019年12月} % 日付
{
  \begin{cvitems} % 業務内容/責任の説明
    \item {知識ライブ配信プラットフォーム「Fun Live」を運営。80回以上の講義を開催し、3万人のユーザーを獲得。PHP、Vue、MySQL、Redisを活用。}
    \item {ソフトウェアコンサルティング事業で50の小規模プロジェクトを完了。収益約300万元、利益約70万元。}
    \item {MianbaoLiveのバックエンド全面リファクタリングを主導。Laravelで再実装し、ヒマラヤとコンテンツ連携。}
    \item {江蘇衛視「最強大脳」微信ミニプログラムのバックエンド全般とフロントエンド半数を開発。}
    \item {HQ Trivia風モバイルクイズアプリ「Chongding Conference」のフルスタック開発を主導。Java、Spring、Redis、Kafkaを活用。}
  \end{cvitems}
}

%---------------------------------------------------------
\cventry
{共同創業者兼フルスタックエンジニア} % 職名
{北京大米娯楽有限公司} % 組織
{中国・北京} % 場所
{2015年11月 - 2016年7月} % 日付
{
  \begin{cvitems} % 業務内容/責任の説明
    \item {コードレビュープラットフォーム「CodeReview」を立ち上げ。3,000ユーザーを獲得。}
    \item {バックエンド全般とフロントエンド半数を開発。PHP、Vue、CodeIgniterを活用。}
  \end{cvitems}
}

%---------------------------------------------------------
\cventry
{ソフトウェアエンジニア} % 職名
{美味書籤(北京)信息技術有限公司} % 組織
{中国・北京} % 場所
{2014年7月 - 2015年11月} % 日付
{
  \begin{cvitems} % 業務内容/責任の説明
    \item {LeanCloud Objective-C SDKとJava SDKの開発に参加。}
    \item {インスタントメッセージングSDKデモアプリ「LeanChat」のiOS/Androidクライアントを開発。}
    \item {iOS SDK、Android SDK、Angularなどの技術を活用。}
  \end{cvitems}
}

%---------------------------------------------------------
\end{cventries}